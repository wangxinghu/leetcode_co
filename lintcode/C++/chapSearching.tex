\chapter{查找}


\section{Search for a Range} %%%%%%%%%%%%%%%%%%%%%%%%%%%%%%
\label{sec:search-for-a-range}


\subsubsection{描述}
Given a sorted array of integers, find the starting and ending position of a given target value.

Your algorithm's runtime complexity must be in the order of $O(\log n)$.

If the target is not found in the array, return \code{[-1, -1]}.

For example,
Given \code{[5, 7, 7, 8, 8, 10]} and target value 8,
return \code{[3, 4]}.


\subsubsection{分析}
已经排好了序,用二分查找。


\subsubsection{使用STL}
\begin{Code}
// LintCode, Search for a Range
// 偷懒的做法,使用STL
// 时间复杂度O(logn),空间复杂度O(1)
class Solution {
public:
    vector<int> searchRange(int A[], int n, int target) {
        const int l = distance(A, lower_bound(A, A + n, target));
        const int u = distance(A, prev(upper_bound(A, A + n, target)));
        if (A[l] != target) // not found
            return vector<int> { -1, -1 };
        else
            return vector<int> { l, u };
    }
};
\end{Code}


\subsubsection{重新实现 lower_bound 和 upper_bound}
\begin{Code}
// LintCode, Search for a Range
// 重新实现 lower_bound 和 upper_bound
// 时间复杂度O(logn),空间复杂度O(1)
class Solution {
public:
    vector<int> searchRange (int A[], int n, int target) {
        auto lower = lower_bound(A, A + n, target);
        auto uppper = upper_bound(lower, A + n, target);

        if (lower == A + n || *lower != target)
            return vector<int> { -1, -1 };
        else
            return vector<int> {distance(A, lower), distance(A, prev(uppper))};
    }

    template<typename ForwardIterator, typename T>
    ForwardIterator lower_bound (ForwardIterator first,
            ForwardIterator last, T value) {
        while (first != last) {
            auto mid = next(first, distance(first, last) / 2);

            if (value > *mid)   first = ++mid;
            else                last = mid;
        }

        return first;
    }

    template<typename ForwardIterator, typename T>
    ForwardIterator upper_bound (ForwardIterator first,
            ForwardIterator last, T value) {
        while (first != last) {
            auto mid = next(first, distance (first, last) / 2);

            if (value >= *mid)   first = ++mid;  // 与 lower_bound 仅此不同
            else                 last = mid;
        }

        return first;
    }
};
\end{Code}

\subsubsection{相关题目}
\begindot
\item Search Insert Position, 见 \S \ref{sec:search-insert-position}
\myenddot


\section{Search Insert Position} %%%%%%%%%%%%%%%%%%%%%%%%%%%%%%
\label{sec:search-insert-position}


\subsubsection{描述}
Given a sorted array and a target value, return the index if the target is found. If not, return the index where it would be if it were inserted in order.

You may assume no duplicates in the array.

Here are few examples.
\begin{Code}
[1,3,5,6], 5 → 2
[1,3,5,6], 2 → 1
[1,3,5,6], 7 → 4
[1,3,5,6], 0 → 0
\end{Code}


\subsubsection{分析}
即\fn{std::lower_bound()}。


\subsubsection{代码}
\begin{Code}
// LintCode, Search Insert Position
// 重新实现 lower_bound
// 时间复杂度O(logn),空间复杂度O(1)
class Solution {
public:
    int searchInsert(int A[], int n, int target) {
        return lower_bound(A, A + n, target) - A;
    }

    template<typename ForwardIterator, typename T>
    ForwardIterator lower_bound (ForwardIterator first,
            ForwardIterator last, T value) {
        while (first != last) {
            auto mid = next(first, distance(first, last) / 2);

            if (value > *mid)   first = ++mid;
            else                last = mid;
        }

        return first;
    }
};
\end{Code}


\subsubsection{相关题目}
\begindot
\item Search for a Range, 见 \S \ref{sec:search-for-a-range}
\myenddot


\section{Search a 2D Matrix} %%%%%%%%%%%%%%%%%%%%%%%%%%%%%%
\label{sec:search-a-2d-matrix}


\subsubsection{描述}
Write an efficient algorithm that searches for a value in an $m \times n$ matrix. This matrix has the following properties:
\begindot
\item Integers in each row are sorted from left to right.
\item The first integer of each row is greater than the last integer of the previous row.
\myenddot

For example, Consider the following matrix:
\begin{Code}
[
  [1,   3,  5,  7],
  [10, 11, 16, 20],
  [23, 30, 34, 50]
]
\end{Code}
Given \fn{target = 3}, return true.


\subsubsection{分析}
二分查找。


\subsubsection{代码}
\begin{Code}
// LintCode, Search a 2D Matrix
// 时间复杂度O(logn),空间复杂度O(1)
class Solution {
public:
    bool searchMatrix(const vector<vector<int>>& matrix, int target) {
        if (matrix.empty()) return false;
        const size_t  m = matrix.size();
        const size_t n = matrix.front().size();

        int first = 0;
        int last = m * n;

        while (first < last) {
            int mid = first + (last - first) / 2;
            int value = matrix[mid / n][mid % n];

            if (value == target)
                return true;
            else if (value < target)
                first = mid + 1;
            else
                last = mid;
        }

        return false;
    }
};
\end{Code}


\subsubsection{相关题目}
\begindot
\item 无
\myenddot
